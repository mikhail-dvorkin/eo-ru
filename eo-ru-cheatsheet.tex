\documentclass{article}
\usepackage[a4paper,margin=2em]{geometry}
\usepackage{polyglossia}
\usepackage{array}
\usepackage{multirow}
\usepackage{tabularx}
\usepackage[table]{xcolor}
\usepackage{hyperref}
\setdefaultlanguage{russian}
\PolyglossiaSetup{russian}{indentfirst=false}
\setmainfont[Mapping=tex-text]{CMU Serif}
\pagestyle{empty}

\def\b#1{\textbf{#1}}
\def\g{\cellcolor{gray!25}}
\setlength\tabcolsep{3pt}

\begin{document}
\begin{center}
\begin{tabular}{l}
\multicolumn{1}{r}{
\begin{tabular}{c}
{\Huge Шпаргалка по эсперанто} \\
{\small Elias N. Jaquez, Krescentja, \href{http://dvorkin.me/}{Mikhail Dvorkin}} \\
{\small \href{https://github.com/mikhail-dvorkin/eo-ru}{github.com/mikhail-dvorkin/eo-ru}} \\
\end{tabular}
\qquad
%\section{Буквы}
\begin{tabular}{|c|c|c|c|c|c|c|c|c|c|c|}
\hline
\vspace*{-0.3em}\b{с} & \b{ĉ} & \b{g} & \b{ĝ} & \b{h} & \b{ĥ} & \b{j} & \b{ĵ} & \b{ŝ} & \b{ŭ} & q, w, x, y \\
ц & ч & г & дж & h & х & й & ж & ш & w & --- \\
\hline
\multicolumn{9}{l}{ударение: предпосл. слог} & \multicolumn{2}{l}{{\tiny ~~$^\nwarrow$~}согласная} \\
\end{tabular}}
\vspace{0.2em}\\

%\section{Окончания}
\begin{tabular}{|r>{\bfseries}l|}
\hline
существит. & -o \\
прилагат. & -a \\
наречие & -e \\
множ.\,число & -j \\
\vspace*{-0.3em}вин.\,падеж & \multirow{2}{*}{\vspace*{0.3em}-n} \\
направление & \\
\hline
\end{tabular}
\quad
%\section{Глаголы}
\begin{tabular}{|cc|c|c|c|c|c|c|}
\hline
\multirow{2}{*}{глагол} & &\g &\g изъяв. &\multicolumn{2}{c|}{\g активное} & \multicolumn{2}{c|}{\g пассивное} \\
\cline{5-8}
& & \g время & \g накл. & \g прич. & \g деепр. & \g прич. & \g деепр. \\
\hline
инфитинив & \b{-i} & \g прош. & \b{-is} & \b{-inta} & \b{-inte} & \b{-ita} & \b{-ite} \\
\cline{3-8}
императив & \b{-u} & \g наст. & \b{-as} & \b{-anta} & \b{-ante} & \b{-ata} & \b{-ate} \\
\cline{3-8}
условное & \b{-us} & \g будущ. & \b{-os} & \b{-onta} & \b{-onte} & \b{-ota} & \b{-ote} \\
\hline
\end{tabular}
\,
%\section{Предлоги и падеж}
\begin{tabular}{ll}
\b{ĵeti tablon} ---\vspace*{-0.3em} &
\multirow{4}{*}{\hspace{-0.75em}\vspace*{1em}$\left. \rule{0pt}{2.3em} \right\}$\,обычно}\\
~~кидать стол\\
\b{ĵeti sur tablo} ---\vspace*{-0.3em}\\
~~кидать на столе\\
\multicolumn{2}{l}{\b{ĵeti sur tablon} ---}\vspace*{-0.3em}\\
\multicolumn{2}{l}{~~кидать на стол [направл.]}\\
\end{tabular}
\vspace{0.2em}\\

%\section{Личные местоимения}
\begin{tabular}{|c|cc|cc|}
\hline
\g & \multicolumn{2}{c|}{\g Един.} & \multicolumn{2}{c|}{\g Множ.} \\
\hline
\g 1 & ~\b{mi}\hspace{-0.5em} & я & ~\b{ni} & мы \\
\hline
\g 2 & \multicolumn{2}{c}{\b{vi}} & \multicolumn{2}{c|}{ты, вы} \\
\hline
\g & \b{li} & он & & \\
\cline{2-3}
\g 3 & \b{ŝi} & она & ~\b{ili} & они \\
\cline{2-3}
\g & \b{ĝi} & оно & & \\	 
\hline
\g $\nearrow$ & \multicolumn{2}{c}{\b{oni}} & \multicolumn{2}{c|}{[безлич.]} \\
\hline
\g $\leftarrow$ & \multicolumn{2}{c}{\b{si}} & \multicolumn{2}{c|}{себя} \\
\hline
\end{tabular}\,
%\section{Аффиксы}
\begin{tabular}{>{\bfseries}rl}
-aĉ- & никчёмность \\
-ad- & длительность \\
-aĵ- & вещь \\
-an- & участник \\
-ar- & совокупность \\
bo- & по браку \\
ĉef- & главный \\
-ĉjo$^*$ & \scalebox{.75}[1.0]{уменьш.-ласк. (м)} \\
dis- & разъединение \\
-ebl- & возможность \\
\end{tabular}
\hspace{-0.75em}
\begin{tabular}{>{\bfseries}rl}
-ec- & свойство \\
-eg- & увеличение \\
-ej- & место \\
ek- & мгновенность \\
eks- & бывший \\
-em- & склонность \\
-end- & обязательно \\
-er- & частица \\
-estr- & начальник \\
-et- & уменьшение \\
\end{tabular}
\hspace{-1em}
\begin{tabular}{>{\bfseries}rl}
fi- & презрение \\
ge- & оба пола \\
-id- & потомок \\
-ig- & делать чем-л. \\
-iĝ- & делаться чем-л. \\
-il- & орудие \\
-in- & женщина \\
-ind- & достойный \\
-ing- & футляр \\
-ism- & доктрина \\
\end{tabular}
\hspace{-1em}
\begin{tabular}{>{\bfseries}rl}
-ist- & профессия \\
mal- & антоним \\
mis- & ошибочность \\
-njo$^*$ & \scalebox{.75}[1.0]{уменьш.-ласк. (ж)} \\
pra- & предок \\
re- & обратно/повтор \\
-uj- & контейнер \\
-ul- & персона \\
-um- & [разное] \\
\multicolumn{2}{r}{$^*$\em{к сокращ. корню}} \\
\end{tabular}
\vspace{0.5em}\\

%\section{Местоимения}
\begin{tabular}{|c|c|c|c|c|c|c|c|c|c|}
\hline
\vspace*{-0.3em}\g & \g\b{-a} & \g\b{-al} & \g\b{-am} & \g\b{-e} & \g\b{-el} & \g\b{-es} & \g\b{-o} & \g\b{-om} & \g\b{-u} \\
\g & \g качество & \g причина & \g время & \g место & \g способ & \g принадл. & \g предмет & \g количество & \g лицо/который \\
\hline
\vspace*{-0.3em}\g\b{ki-} & \b{kia} & \b{kial} & \b{kiam} & \b{kie} & \b{kiel} & \b{kies} & \b{kio} & \b{kiom} & \b{kiu} \\
\g\scalebox{.75}[1.0]{вопросит.} & какой & почему & когда & где & как & чей & что & сколько & кто/который \\
\hline
\vspace*{-0.3em}\g\b{ti-} & \b{tia} & \b{tial} & \b{tiam} & \b{tie} & \b{tiel} & \b{ties} & \b{tio} & \b{tiom} & \b{tiu} \\
\g\scalebox{.75}[1.0]{указат.} & такой & потому & тогда & там & так & того & то & столько & тот/то \\
\hline
\vspace*{-0.3em}\g\b{ĉi-} & \b{ĉia} & \b{ĉial} & \b{ĉiam} & \b{ĉie} & \b{ĉiel} & \b{ĉies} & \b{ĉio} & \b{ĉiom} & \b{ĉiu} \\
\g\scalebox{.75}[1.0]{собират.} & всяческий & \scalebox{.75}[1.0]{по\,всякой\,причине} & всегда & везде & \scalebox{.75}[1.0]{по-всякому} & всеобщий & всё & всё & все/всякий \\
\hline
\vspace*{-0.3em}\g\b{i-} & \b{ia} & \b{ial} & \b{iam} & \b{ie} & \b{iel} & \b{ies} & \b{io} & \b{iom} & \b{iu} \\
\g\scalebox{.75}[1.0]{неопредел.} & какой-то & почему-то & когда-то & где-то & как-то & чей-то & что-то & сколько-то & кто-то/какой-то \\
\hline
\vspace*{-0.3em}\g\b{neni-} & \b{nenia} & \b{nenial} & \b{neniam} & \b{nenie} & \b{neniel} & \b{nenies} & \b{nenio} & \b{neniom} & \b{neniu} \\
\g\scalebox{.75}[1.0]{отрицат.} & никакой & беспричинно & никогда & нигде & никак & ничей & ничто & нисколько & никто/никакой \\
\hline
\end{tabular}
\vspace{0.5em}\\

%\section{Служебные части речи}
\begin{tabular}{>{\bfseries}rl}
%adiaŭ & прощай(те)\\
ajn & \textellipsis~бы ни \\
\multirow{2}{*}{al} & к \\
& {}[дат. падеж] \\
almenaŭ & хотя бы \\
ambaŭ & об(а/е) \\
ankaŭ & тоже, также \\
ankoraŭ & ещё \\
anstataŭ & вместо \\
antaŭ & перед, до \\
apenaŭ & едва \\
apud & около \\
aŭ & или \\
baldaŭ & вскоре \\
cis & по эту сторону \\
ĉar & потому что \\
ĉe & у, при \\
\end{tabular}
\hspace{-3em}
\begin{tabular}{>{\bfseries}rl}
ĉi (ti-) & {}[близость] \\
ĉirkaŭ & вокруг \\
ĉu & ли, вопр. част. \\
da & {}[род. падеж: мера] \\
dank' al & благодаря \\
\multirow{2}{*}{de} & от, из, с \\
& {}[род. падеж] \\
do & итак, же \\
dum & пока, в течение \\
eĉ & даже \\
%ek & начали, айда \\
ekster & вне \\
el & из \\
en & в \\
%fi & фи, фу \\
for & прочь, долой \\
ĝis & пока не, до \\
%ha & ха, ах (удивл.) \\
%hieraŭ & вчера \\
%ho & ох \\
%hodiaŭ & сегодня \\
%hu & ух, ой (испуг) \\
inter & между \\
\end{tabular}
\hspace{-3em}
\begin{tabular}{>{\bfseries}rl}
ja & ведь, же \\
jam & уже \\
je & {}[разное] \\
jen & вот \\
jes & да \\
ju~\textellipsis~des & чем~\textellipsis~тем \\
ĵus & только что \\
%k.a./k.c./k.s. & и прочие \\
kaj & и \\
ke & что (союз) \\
kontraŭ & против \\
krom & кроме \\
k.\,t.\,p. & и так далее \\
kun & с \\
kvankam & хотя \\
kvazaŭ & как будто \\
la & {}[артикль] \\
\end{tabular}
\hspace{-3em}
\begin{tabular}{>{\bfseries}rl}
laŭ & согласно \\
malgraŭ & вопреки \\
mem & сам(а/о) \\
%morgaŭ & завтра \\
ne & нет, не \\
nek & ни \\
%nu & ну \\
nun & сейчас \\
nur & только \\
ol & чем \\
\multirow{2}{*}{per} & с помощью \\
& {}[твор. падеж] \\
plej & самый \\
pli & более \\
plu & дальше, ещё \\
po & по (столько) \\
por & для, чтобы \\
post & после, за \\
\end{tabular}
\hspace{-2em}
\begin{tabular}{>{\bfseries}rl}
preskaŭ & почти \\
preter & мимо \\
pri & o(б), про \\
pro & из-за \\
se & если \\
sed & но \\
sen & без \\
sub & под \\
super & над \\
sur & на \\
tamen & однако \\
tra & сквозь \\
trans & через \\
tre & очень \\
tro & слишком \\
tuj & сразу \\
%ve & увы \\
\end{tabular}
\vspace{0.5em}\\

%\section{Числительные}
\begin{tabular}{|c|}
\hline
\begin{tabular}{c|c|c|c|c|c|c|c|c|c|c|c|c|c|c|c|c|c}
\vspace*{-0.3em}\hspace{-0.3em}$-$ & $+$ & 0 & 1 & 2 & 3 & 4 & 5 & 6 & 7 & 8 & 9 & 10 & 100 & 1000 & $10^6$ & $10^9$ & $10^{6x}$\hspace{-0.3em} \\
\hspace{-0.3em}\b{minus} & \b{plus} & \b{nul} & \b{unu} & \b{du} & \b{tri} & \b{kvar} & \b{kvin} & \b{ses} & \b{sep} & \b{ok} & \b{naŭ} & \b{dek} & \b{cent} & \b{mil} & \b{miliono} & \b{miliardo} & $x$-\b{iliono}\hspace{-0.3em} \\
\end{tabular}\\
\hline
\begin{tabular}{c|c|c|c|c|c|c|c|c|c}
\vspace*{-0.3em}\hspace{-0.3em}\b{trio} & \b{tria} & \b{triono} & \b{trifoje} & \b{triafoje} & \b{triobla} & \b{triobligi} & \b{triope} & \b{trie} & \b{po tri}\hspace{-0.7em} \\
\hspace{-0.5em}тройка, троица, трио & третий & треть & трижды & в третий раз & тройной & утроить & втроём  & в-третьих & по три\hspace{-0.7em} \\
\end{tabular}\\
\hline
\end{tabular}
\vspace{0.5em}\\

%\section{Ссылки}
\multicolumn{1}{c}{
\href{http://eoru.ru/}{eoru.ru}, \href{https://t.me/eoru_bot}{@eoru\texttt{\_}bot}
\quad \href{https://kono.be/vivo}{kono.be/vivo}
\quad \href{https://duolingo.com/course/eo/en}{duolingo.com/course/eo/en}
\quad \href{https://lernu.net/}{lernu.net}
\quad \href{https://verkoj.com/}{verkoj.com}
\quad \href{https://eo.wikipedia.org/}{eo.wikipedia.org}
}
\\
\end{tabular}
\end{center}
\end{document}
