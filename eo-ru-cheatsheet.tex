\documentclass{article}
\usepackage[a3paper,margin=1em]{geometry}
\usepackage{polyglossia}
\usepackage{array}
\usepackage{tabularx}
\setdefaultlanguage{russian}
\PolyglossiaSetup{russian}{indentfirst=false}
\setmainfont[Mapping=tex-text]{CMU Serif}
\pagestyle{empty}

\def\b#1{\textbf{#1}}

\begin{document}

%\section{Буквы}
\begin{tabular}{ccccccccccc}
\b{с} & \b{ĉ} & \b{g} & \b{ĝ} & \b{h} & \b{ĥ} & \b{j} & \b{ĵ} & \b{ŝ} & \b{ŭ} & q, w, x, y \\
ц & ч & г & дж & h & х & й & ж & ш & w & --- \\
\end{tabular}

%\section{Окончания}
\begin{tabular}{r>{\bfseries}l}
существ. & -o \\
прилаг. & -a \\
наречие & -e \\
инфинитив & -i \\
множ. число & -j \\
винит. падеж & -n \\
направление & -n \\
\end{tabular}
%\section{Аффиксы}
\begin{tabular}{>{\bfseries}rl}
-aĉ- & презрение \\
-ad- & длительность \\
-aĵ- & вещь \\
-an- & участник \\
-ar- & совокупность \\
bo- & по браку \\
ĉef- & главный \\
-ĉjo* & уменьш.-ласк. (м) \\
dis- & разъединение \\
\end{tabular}
\begin{tabular}{>{\bfseries}rl}
-ebl- & возможность \\
-ec- & свойство \\
-eg- & увеличение \\
-ej- & место \\
ek- & мгновенность \\
eks- & бывший \\
-em- & склонность \\
-end- & обязательность \\
-er- & частица \\
-estr- & начальник \\
\end{tabular}
\begin{tabular}{>{\bfseries}rl}
-et- & уменьшение \\
fi- & противность \\
ge- & оба пола \\
-id- & потомок \\
-ig- & делать чем-л. \\
-iĝ- & делаться чем-л. \\
-il- & орудие \\
-in- & женщина \\
-ind- & достойный \\
-ing- & футляр \\
\end{tabular}
\begin{tabular}{>{\bfseries}rl}
-ism- & доктрина \\
-ist- & профессия \\
mal- & антоним \\
mis- & ошибочность \\
-njo* & уменьш.-ласк. (ж) \\
pra- & предок \\
re- & повторение \\
-uj- & контейнер \\
-ul- & персона \\
-um- & [разное] \\
\end{tabular}

%\section{Местоимения}
\begin{tabular}{lccccccccc}
 & \b{-a} & \b{-al} & \b{-am} & \b{-e} & \b{-el} & \b{-es} & \b{-o} & \b{-om} & \b{-u} \\
Категория: & качество & причина & время & место & способ & принадл. & предмет & количество & который/индивид \\
\b{ki-} & \b{kia} & \b{kial} & \b{kiam} & \b{kie} & \b{kiel} & \b{kies} & \b{kio} & \b{kiom} & \b{kiu} \\
вопрос. & какой & почему & когда & где & как & чей & что & сколько & кто/который \\
\b{ti-} & \b{tia} & \b{tial} & \b{tiam} & \b{tie} & \b{tiel} & \b{ties} & \b{tio} & \b{tiom} & \b{tiu} \\
указат. & такой & потому & тогда & там & так & того & то & столько & то/тот \\
\b{ĉi-} & \b{ĉia} & \b{ĉial} & \b{ĉiam} & \b{ĉie} & \b{ĉiel} & \b{ĉies} & \b{ĉio} & \b{ĉiom} & \b{ĉiu} \\
определит. & всякий & по всяк. прич. & всегда & везде & по-всякому & прин. всем & всё & всё & все/всякий \\
\b{i-} & \b{ia} & \b{ial} & \b{iam} & \b{ie} & \b{iel} & \b{ies} & \b{io} & \b{iom} & \b{iu} \\
неопредел. & какой-то & почему-то & когда-то & где-то & как-то & чей-то & что-то & сколько-то & кто-то/какой-то \\
\b{neni-} & \b{nenia} & \b{nenial} & \b{neniam} & \b{nenie} & \b{neniel} & \b{nenies} & \b{nenio} & \b{neniom} & \b{neniu} \\
отрицат. & никакой & нипочему & никогда & нигде & никак & ничей & ничто & нисколько & никто/никакой \\
\end{tabular}

%\section{Числительные}
\begin{tabular}{cccccccccccccccc}
0 & 1 & 2 & 3 & 4 & 5 & 6 & 7 & 8 & 9 & 10 & 100 & 1000 & $10^6$ & $10^9$ & $10^{6x}$ \\
\b{nul} & \b{unu} & \b{du} & \b{tri} & \b{kvar} & \b{kvin} & \b{ses} & \b{sep} & \b{ok} & \b{naŭ} & \b{dek} & \b{cent} & \b{mil} & \b{miliono} & \b{miliardo} & $x$-\b{iliono} \\
\end{tabular}

\begin{tabular}{ccccccccc}
\b{trio} & \b{tria} & \b{triono} & \b{trifoje} & \b{triafoje} & \b{triobla} & \b{triope} & \b{trie} & \b{po tri} \\
тройка/троица/трио & третий & треть & трижды & в третий раз & тройной & втроём  & в-третьих & по три \\ 
\end{tabular}

%\section{Служебные части речи}
\begin{tabular}{>{\bfseries}rl}
ajn & - бы ни \\
al & к \\
~ & {}[дат. падеж] \\
almenaŭ & хотя бы \\
ambaŭ & об(а/е) \\
ankaŭ & тоже, также \\
ankoraŭ & ещё \\
anstataŭ & вместо \\
antaŭ & перед, до \\
apenaŭ & едва \\
apud & около \\
aŭ & или \\
baldaŭ & вскоре \\
ĉar & потому что \\
ĉe & у, при \\
ĉi (ti-) & {}[близость] \\
\end{tabular}
\begin{tabular}{>{\bfseries}rl}
ĉirkaŭ & вокруг \\
ĉu & ли, вопр.ч. \\
da & {}[род. падеж: мера] \\
de & от, из, с \\
~ & {}[род. падеж] \\
do & итак, же \\
dum & в течение \\
eble & возможно \\
eĉ & даже \\
ekster & вне \\
el & из \\
en & в \\
fi & фи, фу \\
for & вдали \\
ĝis & пока не, до \\
inter & между \\
\end{tabular}
\begin{tabular}{>{\bfseries}rl}
ja & ведь, же \\
jam & уже \\
je & {}[разное] \\
jen & вот \\
jes & да \\
ju/des & чем/тем \\
ĵus & только что \\
kaj & и \\
ke & что (союз) \\
kontraŭ & против \\
krom & кроме \\
kun & с \\
kvankam & хотя \\
kvazaŭ & как будто \\
la & {}[артикль] \\
laŭ & согласно \\
\end{tabular}
\begin{tabular}{>{\bfseries}rl}
malgraŭ & вопреки \\
mem & сам(а/о) \\
ne & нет, не \\
nek & ни \\
nu & ну \\
nun & сейчас \\
nur & только \\
ol & чем \\
plej & самый \\
pli & более \\
per & с помощью \\
~ & {}[твор. падеж] \\
plu & дальше, ещё \\
por & для, чтобы \\
post & после \\
preskaŭ & почти \\
\end{tabular}
\begin{tabular}{>{\bfseries}rl}
preter & мимо \\
pri & o(б), про \\
pro & из-за \\
se & если \\
sed & но \\
sen & без \\
sub & под \\
super & над \\
sur & на \\
tamen & однако \\
tiel/kiel & так/как \\
tra & сквозь \\
trans & через \\
tre & очень \\
tro & слишком \\
tuj & сразу \\
\end{tabular}

%\section{Личные местоимения}
\begin{tabular}{cccc}
\multicolumn{2}{c}{Единств.} & \multicolumn{2}{c}{Множеств.} \\
\b{mi} & я & \b{ni} & мы \\
\multicolumn{2}{c}{\b{vi}} & \multicolumn{2}{c}{ты, вы} \\
\b{li} & он & & \\
\b{ŝi} & она & \b{ili} & они \\
\b{ĝi} & оно & & \\	 
\multicolumn{2}{c}{\b{oni}} & \multicolumn{2}{c}{[безлич.]} \\
\multicolumn{2}{c}{\b{si}} & \multicolumn{2}{c}{себя} \\
\end{tabular}
%\section{Глаголы}
\begin{tabular}{cccccc}
время & изъяв. & актив. прич. & пассив. прич. & ~ & ~ \\
наст. & \b{-as} & \b{-anta} & \b{-ata} & инфитинив & \b{-i} \\
прош. & \b{-is} & \b{-inta} & \b{-ita} & императив & \b{-u} \\
будущ. & \b{-os} & \b{-onta} & \b{-ota} & условное & \b{-us} \\
\end{tabular}

\end{document}
